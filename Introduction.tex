\section{Introduction}
\label{sec:intro}
While application protocols such as HTTP have evolved to provide reduced latency and efficient use of network resources \cite{rfc7540}, traffic engineering paradigms such as Software Defined Networking (SDN) have simultaneously emerged to provide better Quality-of-Service (QoS) through flexible routing and centralized network management. Several large-scale production Content Distribution Networks (CDNs) such as Google  \cite{Yap:2017}  %and Facebook \cite{Schlinker:2017} 
have implemented Software-Defined Wide Area Networks (SD-WANs) to efficiently perform application-aware routing at the peering edge. According to Cisco \cite{cisco-17}, downstream application traffic is predicted to account for 82\% of all Internet traffic by 2021. Moreover, the same report predicts that SD-WAN traffic will account for 25\% of all WAN traffic by 2021.

HTTP/2 incorporates several improvements over its predecessor, HTTP/1, which include a) multiplexing several streams into one TCP connection, b) server-push approaches, where content is delivered to a client without explicitly requesting it, and c) header compression for reduced latency. These improvements, particularly stream multiplexing, were devised to reduce page load time such that download requests for embedded objects such as images, video, etc., in a web page can be issued simultaneously (instead of sequentially). Similarly, the QUIC \cite{Langley:SIGCOMM:2017} protocol was introduced as a transport layer candidate for HTTP/2 with one basic difference: QUIC is based on UDP and can thus, be used to implement flexible congestion control as well. As protocols become more versatile to support high-bandwidth applications such as Adaptive Bit-Rate (ABR) video streaming and 360 Virtual Reality (VR), network architectures need to adapt in order to meet the demands of such applications worldwide. More recently, the introduction of flexible switch architectures such as \cite{Bosshart:2014} have paved the way for line-rate processing of application-layer headers \cite{jin2017netcache}.
Our demonstration investigates application-based QoS in centrally controlled networks. In particular, this work leverages the capability of protocol-independent packet processors (P4) \cite{Bosshart:2014} at the edge of the network to define a custom fixed-length application header and further, translate this into a Q-in-Q (802.1ad) tag \cite{IEEE802.1ad:standard} for the core network in order to perform QoS routing/provisioning.
%OpenFlow-based \cite{McKeown:2008}. 

In previous work \cite{acm-mmhttp2}, we demonstrated how HTTP/2-based multiplexing can be used to simultaneously fetch multiple qualities of video segments in order to improve the average bitrate quality and thereby, the Quality-of-Experience (QoE) of a client. In this demonstration, we show how HTTP/2 header information can be translated as a QoS requirement using P4-capable network elements to convert application layer header information into a Q-in-Q tag for differentiated routing via the core network using the Bring-Your-Own-Controller (BYOC) feature \cite{chameleon-byoc} provided by the Chameleon testbed \cite{Mambretti:2015}. Although we present a simple prototype using ABR streaming as an example, we believe the capabilities of such a system extend far beyond ABR segment retransmissions and can be used to implement systematic integration of Information Centric Networks (ICN) \cite{Ghodsi:2011} with legacy networks and simultaneous transmission of 360 video viewports \cite{hosseini2016adaptive}. %The remainder of this article provides a detailed description of our architecture (Sect. \ref{sec:design}), followed by our demonstration setup in Sect. \ref{sec:setup} and a Conclusion in Sect. \ref{sec:conclusion}.


